\chapter{Polynomials}

\begin{outline}
    \1 A polynomial in variable $x$ is of the form $$f(x) = a_nx^n + a_{n - 1}x^{n - 1} + \dots + a_1x + a_0$$where $x$ is a variable, $n$ is a positive integer and $a_0, a_1, a_2,\dots,a_n$ are constants.
    \1 The exponent of the highest degree term in a polynomial is known as its degree.
    \1 The names and forms of various degree polynomials:\\\\
    \begin{tabular}{|c|l|l|}
        \hline
        Degree & Name of the polynomial & Form of the polynomial \\
        \hline
        0 & Constant polynomial & $f(x) = a$, $a$ is a constant. \\
        1 & Linear polynomial & $f(x) = ax + b, a \ne 0$ \\
        2 & Quadratic polynomial & $f(x) = ax^2 + bx + c, a \ne 0$ \\
        3 & Cubic polynomial & $f(x) = ax^3 + bx^2 + cx + d, a \ne 0$ \\
        4 & Biquadratic polynomial & $f(x) = ax^4 + bx^3 + cx^2 + dx + e, a \ne 0$ \\
        \hline
    \end{tabular}
    \1 If $f(x)$ is a polynomial and $\alpha$ is any real number, then the real number obtained by replacing $x$ by $\alpha$ in $f(x)$ is known as the value of $f(x)$ at $x = \alpha$ and is denoted by $f(\alpha)$.
    \1 A real number $\alpha$ is a zero of a polynomial $f(x)$, if $f(\alpha) = 0$.
    \1 A polynomial of degree $n$ can have at most $n$ real zeroes.
    \1 Geometrically, the zeroes of a polynomial $f(x)$ are the $x$-coordinates of the points where the graph $y = f(x)$ interesects the $x$-axis.
    \1 If $\alpha$ and $\beta$ are the zeroes of a quadratic polynomial $f(x) = ax^2 + bx + c$, then
    \begin{align*}
        \alpha + \beta = -\frac{b}{a} = -\frac{\text{Coefficient of }x}{\text{Coefficient of }x^2} && \alpha\beta = \frac{c}{a} = \frac{\text{Constant term}}{\text{Coefficient of }x^2} 
    \end{align*}
    \1 $f(x) = x^2 - (\alpha + \beta)x + \alpha\beta$ is a quadratic polynomial whose zeroes are $\alpha$ and $\beta$.
\end{outline}
