\chapter{Real Numbers}

\begin{outline}
    \1 \myvspace 
    \begin{theorem}[Euclid's Division Lemma]
        Given two positive integers $a$ and $b$, there exist unique whole numbers satisfying $a = bq + r, 0 \le r < b$.
    \end{theorem}
    \1 \textit{Euclid's division algorithm}: In order to compute the HCF of two positive integers, say $a$ and $b$, with $a > b$, we take the following steps:
    \2 Apply Euclid's division algorithm to $a$ and $b$. and obtain whole numbers $q_1$ and $r_1$ such that $a = bq_1 + r_1, 0 \le r_1 < b$.
    \2 If $r_1 = 0$, $b$ is the HCF of $a$ and $b$.
    \2 If $r_1 \ne 0$, apply Euclid's division lemma to $b$ and $r_1$ and obtain whole numbers $q_2$ and $r_2$ such that $b = r_1q_2 + r_2$.
    \2 If $r_2 = 0$, then $r_1$ is the HCF of $a$ and $b$.
    \2 If $r_2 \ne 0$, then apply Euclid's division lemma to $r_1$ and $r_2$ and continue the above process until the remainder $r_n$ is zero. The divisor at this stage, i.e. $r_{n - 1}$, or the non-zero remainder at the previous stage, is the HCF of a and b. \vfill
    \1 \myvspace
    \begin{theorem}[The Fundamental Theorem of Arithmetic]
        Every composite number can be expressed as a product of primes, and this factorisation is unique apart from the order in which the prime factors occur.
    \end{theorem}
    \1 Every composite number can be uniquely expressed as a product of powers of primes in ascending or descending order.
    \1 The HCF of two or more numbers is the product of the smallest powers of each of the numbers' common prime factors.
    \1 The LCM of two or more numbers is the product of the greatest powers of each of the numbers' prime factors.
    \1 For any two positive integers $a$ and $b$, $ab$ = $\text{HCF}(a, b) \times \text{LCM}(a, b)$ \newpage
    \1 \myvspace 
    \begin{theorem}
        Let $a$ be a positive integer and $p$ be a prime number. If $p$ divides $a^2$, then $p$ divides $a$. \label{thm:prime}
    \end{theorem}
    \1 If $p$ is a prime number, then $\sqrt{p}$ is an irrational number.
    \begin{proof}
        Assume, to the contrary, that $\sqrt{p}$ is a rational number and express it thus:
        $$\sqrt{p} = \frac{a}{b}$$
        where $a$ and $b$ are co-prime integers and $q \ne 0$. Next, isolate $a$ and square the equation:
        \begin{align}
            b\sqrt{p} &= a \\
            pb^2 &= a^2 \label{eq:irr}
        \end{align}
        Since $p$ divides $a^2$, $p$ divides $a$ (as per Theorem \ref{thm:prime}). Therefore,
        $$a = pc$$
        for some integer $c$. Substitute for $a$ in Equation \ref{eq:irr}:
        \begin{align*}
            pb^2 &= (pc)^2 \\
            pb^2 &= p^2c^2 \\
            b^2 &= pc^2
        \end{align*}
        Since $p$ divides $b^2$, $p$ divides $b$ (as per Theorem \ref{thm:prime}). $p$ is a common factor of $a$ and $b$ This contradicts the fact that $a$ and $b$ are co-prime. Our assumption that $\sqrt{p}$ is rational is incorrect. Hence, $\sqrt{p}$ is irrational.
    \end{proof}
\end{outline}
