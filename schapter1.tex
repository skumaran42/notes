\chapter{Chemical Reactions and Equations}

\section{Introduction}

\begin{outline}

    \1 A chemical reaction is a process in which new substances with new properties are found.

    \1 Chemical reactions involve the breaking and making of bonds between atoms to produce new substances.

    \1 The substances which take part in a chemical reaction are called reactants.

    \1 The new substances produced as a result of a chemical reaction are called products.

    \1 The properties of the products are completely different from the properties of the reactants.

    \1 When a magnesium ribbon is heated, it burns in air (combining with oxygen) with a dazzling white flame to form a white powder called magnesium oxide. This is an example of a chemical reaction.

    \1 A magnesium ribbon usually has a coating of magnesium oxide on its surface which is formed by the slow action of oxygen of air on it. So the magnesium ribbon is cleaned by rubbing with sandpaper before burning it in air. This is done to remove the protective layer of magnesium oxide from the surface of the ribbon so that it may readily combine with the oxygen of air on heating.

    \1 The important characteristics of chemical reactions are:

    \2 Evolution of a gas,
    \2 Formation of a precipitate,
    \2 Change in colour,
    \2 Change in temperature, and
    \2 Change in state

    Any one of these characteristics can tell us whether a chemical reaction has taken place or not.

\end{outline}

\section{Chemical Equations}

\begin{outline}

    \1 The representation of a chemical reaction with the help of symbols and formulae of the substances involved in it is known as a chemical equation.

    \1 In a chemical equation, the reactants are written in the left hand side and the products -- in the right hand side. They separated by an arrow sign pointing towards the right hand side (\ch{->}). Each individual reactant or product is separated from other reactants or products with plus signs (+).
\end{outline}

\subsection{Balanced and Unbalanced Chemical Equations}

\begin{outline}
    \1 A balanced chemical equation has an equal number of atoms of different elements in the reactants and products. 
    
    \1 An unbalanced or skeletal chemical equation has an unequal number of atoms of one or more elements in the reactants and products.

    \1 It is essential to balance chemical equations in order to satisfy the law of conservation of mass in chemical reactions, i.e. matter can neither be created nor destroyed in a chemical reaction.

    \1 In order to balance a chemical equation, symbols and formulae are multiplied by figures like 2, 3 and 4.

    \1 To make a chemical equation more informative, the physical states of the reactants and products are mentioned along with their symbols and formulae. The gaseous, liquid, aqueous and solid states of reactants and products are represented by the notations (g), (l), (aq) and (s) respectively. The word aqueous (aq) is written if the reactant or product is present as a solution in water.

    \1 Reaction conditions such as temperature, pressure, catalyst, etc. are indicated above and/or below the arrow in the equation.

    \1 Some examples of balanced chemical equations:
    \begin{align*}
        \ch{2 Mg_{(s)} + O2_{(g)} &-> 2 MgO_{(s)}} \\
        \ch{Zn_{(s)} + H2SO4_{(aq)} &-> ZnSO4_{(aq)} + H2_{(g)}} \\
        \ch{3 Fe_{(s)} + 4 H2O_{(g)} &-> Fe3O4_{(s)} + 4 H2_{(g)}} \\
        \ch{CO_{(g)} + 2H2_{(g)} &->[340atm] CH3OH_{(l)}} \\
        \ch{6 CO2_{(aq)} + 12 H2O_{(l)} &->[Sunlight][Chlorophyll] C6H12O6_{(aq)} + 6 O2_{(aq)} + 6 H2O_{(l)}} \\
        \ch{H2_{(g)} + Cl2_{(g)} &-> 2 HCl_{(g)}} \\
        \ch{3 BaCl2_{(aq)} + Al2(SO4)3_{(aq)} &-> 3 BaSO4_{(s)} + 2 AlCl3_{(aq)}} \\
        \ch{2 Na_{(s)} + 2 H2O_{(l)} &-> 2 NaOH_{(aq)} + H2_{(g)}} \\
        \ch{NaOH_{(aq)} + HCl_{(aq)} &-> NaCl_{(aq)} + H2O_{(l)}}
    \end{align*}
\end{outline}

\section{Types of Chemical Reactions}

\subsection{Combination Reactions}

\begin{outline}
    \1 Reactions in which two or more reactants combine to form a single product are called combination reactions. 

    \1 Examples of combination reactions:
    
    \2 Calcium oxide (quick lime) reacts vigorously with water to produce calcium hydroxide (slaked lime), releasing a large amount of heat.\\
    \ch{CaOH_{(s)} + H2O_{(l)} -> Ca(OH)2_{(aq)} + Heat}

    \begin{remark}
        A solution of slaked lime is used for whitewashing walls. Calcium hydroxide reacts slowly with the carbon dioxide in air to form a thin layer of calcium carbonate on the walls. Calcium carbonate is formed after two to three days of whitewashing and gives a shiny finish to the walls. The chemical formula for marble is also \textup{\ch{CaCO3}}.

        \textup{\ch{Ca(OH)2_{(aq)} + CO2_{(g)} -> CaCO3_{(s)} + H2O_{(l)}}}
    \end{remark}

    \2 Burning of coal\\
    \ch{C_{(s)} + O2_{(g)} -> CO2_{(g)}}

    \2 Formation of water from hydrogen and oxygen gases\\
    \ch{2 H2_{(g)} + O2_{(g)} -> 2 H2O_{(l)}}

    \1 Reactions in which heat is released along with the formation of products are called exothermic reactions.

    \1 Examples of exothermic reactions:

    \2 Burning of natural gas\\
    \ch{CH4_{(g)} + 2 O2_{(g)} -> CO2_{(g)} + 2 H2O_{(g)}}
    
    \2 Respiration \\
    \ch{C6H12O6_{(aq)} + 6 O2_{(aq)} -> 6 CO2_{(aq)} + 6 H2O_{(l)} + Energy}

    \2 Decomposition of vegetable matter into compost

    \2 Burning of magnesium ribbon
\end{outline}

\subsection{Decomposition Reactions}

\begin{outline}
    \1 Reactions in which a single reactant breaks down to give simpler products are called decomposition reactions.
    
    \1 Reactions in which energy is absorbed from the surroundings to form products are known as endothermic reactions. For example:

    \ch{Ba(OH)2_{(s)} + 2 NH4Cl_{(s)} -> BaCl2_{(s)} + 2 NH3_{(g)} + 2 H2_{(g)}}

    \1 All decomposition reactions are endothermic

    \1 Primarily, there are three types of decomposition reactions:
    
    \2 Thermal decomposition: energy is supplied to the reactants in the form of heat
    \2 Electrolytic decomposition: energy is supplied in the form of electricity
    \2 Photolytic decomposition: energy is supplied in the form of light

    \1 Examples of decomposition reactions:

    \2 Decomposition of ferrous sulphate \\
    \ch{2 FeSO4_{(s)} ->[Heat] Fe2O3_{(s)} + SO2_{(g)} + SO3_{(g)}}

    The green colour of the ferrous sulphate crystals (\ch{2 FeSO4 * 7 H2O}) changes into white when they become anhydrous (lose water) due to heating. Upon further heating, ferrous sulphate decomposes into ferric oxide, which is reddish-brown in colour.

    \2 Decomposition of lead nitrate \\
    \ch{2 Pb(NO3)2_{(s)} ->[Heat] 2 PbO_{(s)} + NO2_{(g)} + O2_{(g)}}
    
    This involves the emission of brown fumes, which are of nitrogen dioxide.

    \2 Electrolysis of water \\
    \ch{2 H2O_{(l)} ->[Electricity] 2 H2_{(g)} + O2_{(g)}}

    Hydrogen accumulates next to the cathode (negative terminal) while oxygen accumulates next to the anode (positive terminal).

    \2 Decomposition of silver chloride \\
    \ch{2 AgCl_{(s)} ->[Sunlight] 2 Ag_{(s)} + Cl2_{(g)}}

    White silver chloride decomposes into grey silver and chlorine.

    \2 Decomposition of silver bromide \\
    \ch{2 AgBr_{(s)} ->[Sunlight] 2 Ag_{(s)} + Br2_{(g)}}

    The last two reactions are used in black and white photography.
\end{outline}

\subsection{Displacement Reactions}

\begin{outline}
    \1 Reactions in which one element takes the place of another element in a compound are known as displacement reactions.

    \1 In general, a more reactive element displaces a less reactive element from its compound.

    \1 Some examples of displacement reactions:

    \2 \ch{Fe_{(s)} + CuSO4_{(aq)} -> FeSO4_{(aq)} + Cu_{(s)}} \\
    Iron, being more reactive than copper, displaces it from copper sulphate solution. The iron nail becomes brownish in colour because of deposition of copper, and conversion of copper sulphate to iron sulphate makes the blue colour of the solution fade.

    \2 \ch{Zn_{(s)} + CuSO4_{(aq)} -> ZnSO4_{(aq)} + Cu_{(s)}}

    \2 \ch{Pb_{(s)} + CuCl2_{(aq)} -> PbCl2_{(aq)} + Cu_{(s)}}

    Iron and lead are more reactive than copper; they displace it from its compounds.
\end{outline}

\subsection{Double Displacement Reaction}

\begin{outline}

    \1 Reactions in which two compounds react by an exchange of ions to form two new compounds are called double displacement reactions.

    \1 A precipitate is an insoluble solid product of a chemical reaction.

    \1 Reactions involving the formation of a precipitate are known as precipitation reactions.

    \1 Examples of double displacement reactions (which are also precipitation reactions):

    \2 \ch{Na2SO4_{(aq)} + BaCl2_{(aq)} -> BaSO4_{(s)} + 2 NaCl_{(aq)}}

    This involves the formation of a white precipitate, namely, barium chloride.

    \2 \ch{Pb(NO3)2_{(aq)} + 2 KI_{(aq)} -> PbI2_{(s)} + 2 KNO3_{(aq)}}

    This involves the formation of a yellow precipitate, namely, potassium iodide.

    \begin{remark}
        This reaction is also known as the Golden Rain Reaction. 
    \end{remark}
\end{outline}

\subsection{Oxidation and Reduction}

\begin{outline}
   
    \1 Oxidisation can be defined as:

    \2 The addition of oxygen to a substance
    \2 The removal of hydrogen from a substance

    \1 Reduction can be defined as:

    \2 The addition of hydrogen to a substance
    \2 The removal of oxygen to a substance

    \1 Oxidisation and reduction are inverse processes of each other. They occur together.

    \1 An oxidising agent can be defined as:

    \2 A substance which gives oxygen for oxidation
    \2 A substance which removes hydrogen

    \1 A reducing agent can be defined as:

    \2 A substance which gives hydrogen for reduction
    \2 A substance which removes oxygen
    
    \1 Reactions that involve oxidation and reduction are called oxidation-reduction reactions or redox reactions.

    \1 Examples of redox reactions:

    \2 \ch{2 Cu_{(s)} + O2_{(g)} ->[Heat] 2 CuO_{(g)}}
    The surface of copper powder becomes coated with black copper oxide upon heating. Copper is oxidised to copper oxide while oxygen is reduced to copper oxide.

    \2 \ch{CuO_{(s)} + H2_{(g)} ->[Heat] Cu_{(s)} + H2O_{(g)}}\\
    Black copper oxide turns brown due to formation of copper if hydrogen gas is passed over copper oxide; this is the reverse reaction of the earlier one. Copper oxide is reduced to copper and hydrogen is oxidised to water vapour.

    \2 \ch{ZnO_{(s)} + C_{(s)} -> Zn_{(g)} + CO_{(g)}}\\
    Zinc oxide is reduced to zinc vapour and carbon is oxidised to carbon monoxide.

    \2 \ch{MnO2_{(s)} + 4 HCl_{(aq)} -> MnCl2_{(aq)} + 2 H2O_{(l)} + Cl2_{(g)}}\\
    Dilute hydrochloric acid is oxidised to chlorine gas and manganese dioxide is reduced to manganese chloride.

\end{outline}

\section{Effects of Oxidation Reactions in Everyday Life}

\subsection{Corrosion}

\begin{outline}
    \1 Corrosion is the process in which metals are eaten up gradually by the action of air, moisture or a chemical (such as an acid) on their surface.

    \1 A main cause of corrosion is the oxidation of metals by atmospheric oxygen. 

    \1 When iron is exposed to damp air for a considerable time, it gets covered with a red-brown flaky substance called 'rust'. This is the most common form of corrosion, and moreover, it is a redox reaction.

    \1 Formations of a black coating on silver and a green coating on copper are other examples of corrosion.

    \1 Corrosion causes damage to car bodies, bridges, iron railings, ships and to all objects made of metals, specially those of iron.
\end{outline}

\subsection{Rancidity}

\begin{outline}
    \1 When the fats and oils present in food items get oxidised by atmospheric oxygen, the products of oxidation have an unpleasant smell and taste. The food materials are said to have become 'rancid'.

    \1 Rancidity is the condition produced by aerial oxidation of fats and oils, marked by an unpleasant smell and taste.

    \1 Rancidity can be prevented by:

    \2 Adding antioxidants (substances which prevent oxidation) to foods containing fats and oil.
    \2 Keeping food in air tight containers (this slows down oxidation)
    \2 Packaging foodstuffs containing fats and oil in nitrogen gas.
\end{outline}
