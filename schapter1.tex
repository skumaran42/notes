\chapter{Chemical Reactions and Equations}

\section{Introduction}

\begin{outline}

    \1 A chemical reaction is a process in which new substances with new properties are found.

    \1 Chemical reactions inolve the breaking and making of bonds between atoms to produce new substances.

    \1 The substances which take part in a chemical reaction are called reactants.

    \1 The new substances produced as a result of a chemical reaction are called products.

    \1 The properties of the products are completely different from the properties of the reactants.

    \1 When a magnesium ribbon is heated, it burns in air (combining with oxygen) with a dazzling white flame to form a white powder called magnesium oxide. This is an example of a chemical reaction.

    \1 A magnesium ribbon usually has a coating of magnesium oxide on its surface which is formed by the slow action of oxygen of air on it. So the magnesium ribbon is cleaned by rubbing with sandpaper before burning it in air. This is done to remove the protective layer of magnesium oxide from the surface of the ribbon so that it may readily combine with the oxygen of air on heating.

    \1 The important characteristics of chemical reactions are:

    \2 Evolution of a gas,
    \2 Formation of a precipitate,
    \2 Change in colour,
    \2 Change in temperature, and
    \2 Change in state

    Any one of these characteristics can tell us whether a chemical reaction has taken place or not.

\end{outline}

\section{Chemical Equations}

\begin{outline}

    \1 The representation of a chemical reaction with the help of symbols and formulae of the substances involved in it is known as a chemical equation.

    \1 In a chemical equation, the reactants are written in the left hand side and the products -- in the right hand side. They separated by an arrow sign pointing towards the right hand side (\ch{->}). Each individual reactant or product is separated from other reactants or products with plus signs (+).
\end{outline}

\subsection{Balanced and Unbalanced Chemical Equations}

\begin{outline}
    \1 A balanced chemical equation has an equal number of atoms of different elements in the reactants and products. 
    
    \1 An unbalanced or skeletal chemical equation has an unequal number of atoms of one or more elements in the reactants and products.

    \1 It is essential to balance chemical equations in order to satisfy the law of conservation of mass in chemical reactions, i.e. matter can neither be created nor destroyed in a chemical reaction.

    \1 In order to balance a chemical equation, symbols and formulae are multiplied by figures like 2, 3 and 4.

    \1 To make a chemical equation more informative, the physical states of the reactants and products are mentioned along with their symbols and formulae. The gaseous, liquid, aqueous and solid states of reactants and products are represented by the notations (g), (l), (aq) and (s) respectively. The word aqueous (aq) is written if the reactant or product is present as a solution in water.

    \1 Reaction conditions such as temperature, pressure, catalyst, etc. are indicated above and/or below the arrow in the equation.

    \1 Some examples of balanced chemical equations:
    \begin{align*}
        \ch{2 Mg_{(s)} + O2_{(g)} &-> 2 MgO_{(s)}} \\
        \ch{Zn_{(s)} + H2SO4_{(aq)} &-> ZnSO4_{(aq)} + H2_{(g)}} \\
        \ch{3 Fe_{(s)} + 4 H2O_{(g)} &-> Fe3O4_{(s)} + 4 H2_{(g)}} \\
        \ch{CO_{(g)} + 2H2_{(g)} &->[340atm] CH3OH_{(l)}} \\
        \ch{6 CO2_{(aq)} + 12 H2O_{(l)} &->[Sunlight][Chlorophyll] C6H12O6_{(aq)} + 6 O2_{(aq)} + 6 H2O_{(l)}} \\
        \ch{H2_{(g)} + Cl2_{(g)} &-> 2 HCl_{(g)}} \\
        \ch{3 BaCl2_{(aq)} + Al2(SO4)3_{(aq)} &-> 3 BaSO4_{(s)} + 2 AlCl3_{(aq)}} \\
        \ch{2 Na_{(s)} + 2 H2O_{(l)} &-> 2 NaOH_{(aq)} + H2_{(g)}} \\
        \ch{NaOH_{(aq)} + HCl_{(aq)} &-> NaCl_{(aq)} + H2O_{(l)}}
    \end{align*}
\end{outline}

\section{Types of Chemical Reactions}

